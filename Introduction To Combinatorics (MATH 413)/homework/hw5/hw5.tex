\documentclass[12pt]{exam}
\usepackage{amsthm}
\usepackage{libertine}
\usepackage[utf8]{inputenc}
\usepackage[margin=1in]{geometry}
\usepackage{amsmath,amssymb}
\usepackage{multicol}
\usepackage[shortlabels]{enumitem}
\usepackage{siunitx}
\usepackage{cancel}
\usepackage{graphicx}
\usepackage{pgfplots}
\usepackage{listings}
\usepackage{tikz}


\pgfplotsset{width=10cm,compat=1.9}
\usepgfplotslibrary{external}
\tikzexternalize

\newcommand{\class}{MATH 413} % This is the name of the course 
\newcommand{\examnum}{Homework 5} % This is the name of the assignment
\newcommand{\examdate}{10/10/2022} % This is the due date
\newcommand{\timelimit}{}


\begin{document}
\pagestyle{plain}
\thispagestyle{empty}

\noindent
\begin{tabular*}{\textwidth}{l @{\extracolsep{\fill}} r @{\extracolsep{6pt}} l}
\textbf{\class} & \textbf{Name:} & \textit{Amit Sawhney}\\ %Your name here instead, obviously 
\textbf{\examnum} &&\\
\textbf{\examdate} &&\\
\end{tabular*}\\
\rule[2ex]{\textwidth}{2pt}
% ---

\begin{enumerate}
    \item Question 11
    
    Let $S$ be a subset with three distinguished elements $a, b, c$. The number of $k$ subsets in $S$ is $\binom{n}{k}$. Additionally, the number of $k$ subsets in $S\setminus \{ a, b, c\}$ is $\binom{n - 3}{k}$. Clearly, the number of subsets that contain at least $\{a, b, c\}$ is $\binom{n}{k} - \binom{n - 3}{k}$ which is the LHS. Decompose $S$ into three types of subsets. The first type of subset only contains $c$, The second type of subset can contain $b$ but not $a$. And the the third type of subset can contain $a$. For type 1, there are $\binom{n - 3}{k - 1}$ subsets, For type 2, there are $\binom{n - 2}{k - 1}$ subsets. For type 3, there are $\binom{n - 1}{k - 1}$ subsets. When added together, this is the RHS. 
    \item Question 12
    
    
    
    \item Question 16
    
    Consider the formula for binomial expansion: $$(1 + x)^n = \sum_{k = 0}^{n} \binom{n}{k}x^k$$
    
    Integrate both sides of this equation from 0 to 1 over $x$ as follows:
    
    LHS: 
    $$\int_{0}^{1}(1+x)^n = \frac{(1+x)^{n + 1}}{n + 1} \Bigg |_0^1 = \frac{(1 + 1)^{n + 1}}{n + 1} - \frac{(1)^{n + 1}}{n + 1} = \frac{2^{n + 1} - 1}{n + 1}$$
    
    RHS:
    $$
    \int_{0}^{1} \sum_{k = 0}^{n} \binom{n}{k}x^k = \sum_{k = 0}^{n} \binom{n}{k}\int_{0}^{1}x^k = \sum_{k = 0}^{n} \binom{n}{k} \Bigg ( \frac{x^{k + 1}}{k + 1}\Bigg |_0^1 \Bigg ) = \sum_{k = 0}^{n} \binom{n}{k} \Bigg (\frac{(1)^{k + 1}}{k + 1} - \frac{(0)^{k + 1}}{k + 1} \Bigg )
    $$
    $$
    = \sum_{k = 0}^{n} \binom{n}{k} \Bigg ( \frac{1}{k + 1} \Bigg ) = \binom{n}{0} + \frac{1}{2} \cdot \binom{n}{1} + \frac{1}{3} \cdot \binom{n}{2} + \cdots \binom{n}{n} \cdot \frac{1}{n + 1}
    $$
    
    Clearly, this proves the identity. 
    
    \item Question 27
    
    Consider $S$ to be a set of all of the ways to pick a team of size $k$ from $n$ people with 1 director or 2 directors. 
    
    \begin{enumerate}
        \item We can count this by picking the teams first and picking directors from the team. 
        
        Construct an algorithm that picks a team with $k$ people. For each $k$ member team, there are $k$ ways to pick a singular director and $k(k-1)$ ways to pick co-directors. Considering all possible sizes of $k$, there are:
        
        $$\sum_{k = 1}^{n} \binom{n}{k} \cdot k(k -1) +k = \sum_{k = 1}^n \binom{n}{k}\cdot k^2 -k + k = \sum_{k =1}^n k^2\binom{n}{k}$$
        
        ways to do this. This is the RHS.
        
        \item We can count this by picking the directors first and then picking who is on each team of directors.
        
        There are $n$ ways to choose a singular director. For the remaining people, they are either on the directors team or not, so there are $2^{n - 1}$ possibilities for the remaining people. If there are 2 directors, there are $n(n - 1)$ ways to form 2 co-directors with $n$ people. From the remaining people they, are either on the directors team or not, so there are $2^{n - 2}$ ways to do this for the remaining $n - 2$ people.
        
        In total, $$n2^{n - 1} + n(n -1)2^{n - 2} = n(n + 1)2^{n -2}$$ ways to choose these teams, which is the LHS.
        
    \end{enumerate}
    
    \item Question 28
    
    For clarity, this question can be written as,
    
    $$\sum_{k = 1}^{n} k \binom{n}{k} \binom{n}{k} = n \binom{2n - 1}{n - 1}$$
    
    Imagine there are $2n$ people trying out for a football team where only juniors and seniors can try out. The coach must select $n$ people and 1 captain (where only a senior can be captain). Let $S$ be the set of teams with a distinct captain that can be formed. 
    
\begin{enumerate}
    \item LHS
    
    Decompose this set $S$ into the set of $n$ juniors and $n$ seniors. For each size $k$ from $0$ to $n$, we can select $k$ juniors and then $k$ seniors. From this selection, there are $k$ choices for the captain. Mathematically, there are $\binom{n}{k}$ possibilities for the juniors, then $\binom{n}{k}$ possibilities for the seniors. Lastly, there are $k$ possibilities for the captain as there are $k$ seniors, so $$\sum_{k = 1}^{n} \binom{n}{k} \binom{n}{k} k$$
    
    \item RHS
    
    Once again, decompose this set into $n$ juniors and $n$ seniors. There are $n$ seniors and so there are $n$ choices for the captain. After the captain is selected, select the remaining $n - 1$ people. Mathematically, there are $\binom{2n - 1}{n - 1}$ ways to choose the remaining people. So, $$n \binom{2n - 1}{n - 1}$$ ways to count $S$ with this algorithm. This is the RHS. 
    
\end{enumerate}
\end{enumerate}


\end{document}

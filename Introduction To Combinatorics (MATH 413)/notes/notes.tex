\documentclass{report}

\input{preamble}
\input{macros}
\input{letterfonts}

\title{\Huge{MATH 413}\\Introduction to Combinatorics}
\author{\huge{Amit Sawhney}}
\date{}

\begin{document}

\maketitle
\newpage% or \cleardoublepage
% \pdfbookmark[<level>]{<title>}{<dest>}
\pdfbookmark[section]{\contentsname}{toc}
\tableofcontents
\pagebreak

\chapter{What is Combinatorics?}

\chapter{Permutations and Combinations}

\section{Lecture 2: Four Basic Counting Principles}

\section{Lecture 3: Permutations and selections of sets I}

\section{Lecture 4: Permutations and selections of sets II: binomial identities}

\section{Lecture 5: Permutations and Combinations of multisets I}

\section{Lecture 6: Permutations and Combinations of multisets II}

\chapter{The Pigeonhole Principle}

\section{Lecture 7: The pigeonhole principle}

\section{Lecture 8: The strong pigeonhole principle}

\section{Lecture 9: Ramsey Theory}

\setcounter{chapter}{4}

\chapter{The Binomial Coefficients}

\section{Lecture 10: Binomial coefficients and the binomial theorem I}

\section{Lecture 11: Binomial coefficients and the binomial theorem II}

\section{Lecture 12: Binomial coefficients and the binomial theorem III}

\chapter{The Inclusion-Exclusion Principle and Applications}

\section{Lecture 13: The Inclusion-Exclusion principle and applications I}

\section{Lecture 14: The Inclusion-Exclusion principle and applications II: Derangements}

\section{Lecture 15: The Inclusion-Exclusion principle and applications II}

\section{Lecture 16: The Inclusion-Exclusion principle and applications IV: Another Forbidden Position Problem}

\chapter{Recurrence Relations and Generating Functions}
\section{Lecture 17: Some Number Sequences}

\ex{Example 1}{
    Consider a configuration of $n$ lines where every two lines
    have a point in common, but no three do. How many regions in the
    plane are there? Give a recurrence. \\

    $$
        a_n = a_{n-1} + n
    $$

    TODO: Give an explanation of why this is true.
}

\ex{Example 2}{
    Give a simple recurrence for dearragements. \\

    $$
        D_n = (n - 1)(D_{n-1} + D_{n-2})
    $$

    TODO: Give an explanation of why this is true. Need to review dearragements
    from previous lecture.
}

\subsubsection*{}
Consider the Fibonacci sequence $f_n = f_{n-1} + f_{n-2}$, where $f_0 = 0$ and $f_1 = 1$.

\dfn{The adjusted Fibonacci sequence: $\hat{F}_n$}{
    This is the number of 1,2 lists of size $n$. In other words,
    consider the number of ways a valet can park A cars (size $1$) and
    B cars (size $2$) in a parking lot of size $n$.

    \begin{equation*}
        \hat{F}_n = \begin{cases}
            1         & \text{if $n = 0$} \\
            f_{n + 1} & \text{otherwise}
        \end{cases}
    \end{equation*}
}

\qs{}{
    Prove
    $$
        \sum_{n=0}^{n} f_{i} = f_{n+2} - 1
    $$
}
\sol{
    \begin{subproof}
        We will prove this by induction on $n$. \\

        \textbf{Base case}: $n = 0$.
        $$
            \sum_{i=0}^{0} f_{i} = f_{0} = 0
        $$

        Similarly,
        $$
            f_{0+2} - 1 = f_{2} - 1 = 1 - 1 = 0
        $$

        So, the base case is true. \\

        \textbf{Inductive Hypothesis}: Assume that the following statement is true for $n = k$.

        $$
            \sum_{i=0}^{k} f_{i} = f_{k+2} - 1
        $$

        \textbf{Indcutive Step}: We will prove that the following statement is true for $n = k + 1$.

        $$
            \sum_{i=0}^{k+1} f_{i} = f_{k+1} + \sum_{i=0}^{k} f_{i} = f_{k+1} + f_{k+2} - 1 = f_{k+3} - 1
        $$

        Therefore, the statement is true for all $n$ by induction.
    \end{subproof}
}

\qs{}{
    Prove
    $$
        1 + \sum_{i=0}^{n} \hat{F}_i = \hat{F}_{n+2}
    $$
}
\sol{
    \begin{subproof}
        We will prove this by induction on $n$. \\

        \textbf{Base case}: $n = 0$.
        $$
            1 + \sum_{i=0}^{0} \hat{F}_i = 1 + \hat{F}_0 = 1 + 1 = 2
        $$
        Similarly,
        $$
            \hat{F}_{0+2} = \hat{F}_{2} = f_{2+1} = f_{3} = 2
        $$
        So, the base case is true. \\

        \textbf{Inductive Hypothesis}: Assume that the following statement is true for $n = k$.
        $$
            1 + \sum_{i=0}^{k} \hat{F}_i = \hat{F}_{k+2}
        $$

        \textbf{Inductive Step}: We will prove that the following statement is true for $n = k + 1$.
        \begin{align*}
            1 + \sum_{i=0}^{k+1} \hat{F}_i & = 1 + \hat{F}_{k+1} + \sum_{i=0}^{k} \hat{F}_i \\
                                           & = \hat{F}_{k+1} + \hat{F}_{k+2}                \\
                                           & = f_{k+2} + f_{k+3}                            \\
                                           & = f_{k+4}                                      \\
                                           & = \hat{F}_{k+3}                                \\
        \end{align*}

        Therefore, the statement is true for all $n$ by induction.
    \end{subproof}

    \qs{}{
        Prove that $f_n$ is even if and only if $n$ is divisible by $3$.
    }
    \sol{
        \begin{subproof}
            Given that $f_0 = 0$, $f_1 = 1$, and $f_2 = 1$, we can see that
            at $n = 3$, $f_3 = 2$, which is even. \\

            This is because the only way to get an even number is to have
            the parity of the two numbers added togethed (odd + odd or even + even)
            be the same. So, $f_4$, must be odd, $f_5$ must be odd and
            $f_6$ must be even. \\

            Given the starting sequence of even, odd, odd. The following sequence
            must always be even, odd, odd, which repeats every $3$ numbers.

            Since the first $n = 0$ is the first number in the sequence, every $n$
            that is divisible by $3$ is even.
        \end{subproof}
    }
    \nt{
        Example problems for later \\

        Guess and prove by induction (you may replace the Fibonnaci 
        number by the adjusted Fibonacci number if it helps you) \\

        \begin{itemize}
            \item $f_1 + f_3 + \cdots + f_{2n - 1} = ?$
            \item $f_0 + f_2 + \cdots + f_{2n} = ?$
            \item $f_0 - f_1 + f_2 - \cdots + (-1)^{n}f_n = ?$
            \item $(f_0)^2 + (f_1)^2 + \cdots + (f_n)^2 = ?$
        \end{itemize}
    }

    \subsubsection*{Obtaining an explicit formula for $f_n$ for linear recurrences}
    \ex{}{
        Consider the Fibonacci sequence $f_n = f_{n-1} + f_{n-2}$, where $f_0 = 0$ and $f_1 = 1$.
        This can be rewritten as a linear recurrence as follows:
        $$
        f_n - f_{n-1} - f_{n-2} = 0
        $$

        We must solve the corresponding characterstic equation. Notice how the largest
        degree lines up with the "largest" case of the recurrence. \\

        $$
        x^2 - x - 1 = 0
        $$

        Let $q_1$ and $q_2$ be the roots of the characteristic equation. \\

        It is potentially relevant to note that the following is a solution space
        of the Fibonacci recurrence (but don't satisfy $f_0$ -- 
        the initial condition): \\

        \begin{equation*}
            \begin{cases}
                q_1^n - q_1^{n - 1} - q_1^{n - 2} = 0 \\
                q_2^n - q_2^{n - 1} - q_2^{n - 2} = 0 \\
            \end{cases}
        \end{equation*}

        The rest of this is based on an ansatz, i.e. 
        we need to make an assumption at the answer and validate it later \\

        $$
        f_n = c_1q_1^n + c_2q_2^n,
        $$

        for \textit{some} $c_1, c_2 \in \mathbb{R}$. \\

        Using the initial conditions of $f_0 = 0$ and $f_1 = 1$, 
        we can solve for $c_1$ and $c_2$.
    }

    \section{Lecture 18: Introduction to ordinary generating series}

    \chapter{Special Counting Sequences}

    \section{Lecture 19: Partition identities}

    \section{Lecture 20: Partition identities (continued)}

    \section{Lecture 21: Exponential generating series}
}

\end{document}
\documentclass[12pt]{exam}
\usepackage{amsthm}
\usepackage{libertine}
\usepackage[utf8]{inputenc}
\usepackage[margin=1in]{geometry}
\usepackage{amsmath,amssymb}
\usepackage{multicol}
\usepackage[shortlabels]{enumitem}
\usepackage{siunitx}
\usepackage{cancel}
\usepackage{graphicx}
\usepackage{pgfplots}
\usepackage{listings}
\usepackage{tikz}


\pgfplotsset{width=10cm,compat=1.9}
\usepgfplotslibrary{external}
\tikzexternalize

\newcommand{\class}{MATH 413} % This is the name of the course 
\newcommand{\examnum}{Homework 3} % This is the name of the assignment
\newcommand{\examdate}{9/14/2022} % This is the due date
\newcommand{\timelimit}{}


\begin{document}
\pagestyle{plain}
\thispagestyle{empty}

\noindent
\begin{tabular*}{\textwidth}{l @{\extracolsep{\fill}} r @{\extracolsep{6pt}} l}
\textbf{\class} & \textbf{Name:} & \textit{Amit Sawhney}\\ %Your name here instead, obviously 
\textbf{\examnum} &&\\
\textbf{\examdate} &&\\
\end{tabular*}\\
\rule[2ex]{\textwidth}{2pt}
% ---




\begin{enumerate} %You can make lists!

\item Hotel Problem

Solution: $301595092259134915263594496.0$

Order of Magnitude: $26$

\begin{lstlisting}[language=Python]
import scipy.special as spys
import numpy as np
import math

NUM_ROOMS = 8
NUM_PEOPLE = 32
NUM_PER_ROOM = 4

dp = np.zeros((NUM_ROOMS + 1, NUM_PEOPLE + 1))

for person in range(0, NUM_PEOPLE + 1):
  # traverse from NUM_ROOMS to 0
  for room in range(NUM_ROOMS, -1, -1):
    if room == NUM_ROOMS and (person >= 0 or person <= 4):
      dp[room][person] = 1
    elif person == 0:
      dp[room][person] = 1
    else:
      for occupant in range(0, NUM_PER_ROOM + 1):
        dp[room][person] += spys.comb(person, occupant) *
        dp[room + 1][person - occupant]
    
total = np.sum(dp[1:])

\end{lstlisting}

Essentially, we are counting how many ways we can put 1 person into 1 room, 2 rooms, 3 rooms, $\dots$, 8 rooms. And then doing the same for 2 people, 3 people, $\dots$, 32 people. By caching the amounts of previous counts using a DP algorithm, we can easily calculate later probabilities for by referencing cached values. 

\item Q57

$$
\frac{\binom{13}{1} \cdot \binom{4}{2} \cdot \binom{12}{3} \cdot \binom{4}{1} \cdot \binom{4}{1} \cdot \binom{4}{1} \cdot }{\binom{52}{5}} = 0.4226
$$

There are clearly $\binom{52}{5}$ ways to select 5 cards from a deck. There are $\binom{13}{1}$ ways to pick which card will have a pair. There are $\binom{4}{2}$ ways to pick what specific cards of that rank will make up the pair. Then there are $\binom{12}{3}$ ways to choose what the remaining three cards will be. And for each of these cards there are $\binom{4}{1}$ ways to pick which card each will be. 

\item Q60

For the first question, 

$$
\frac{\binom{9 + 6 - 1}{9}}{\binom{15 + 6 - 1}{15}} = 0.1291
$$

This makes sense because there are $\binom{15 + 6 - 1}{15}$ ways to 15 bagels from 6 types of bagels and there are $\binom{9 + 6 - 1}{9}$ ways to select 9 free degrees of bagel from 6 types of bagel (which ensures that there is 1 type of each bagel). 

$$
\frac{\binom{7 + 6 - 1}{7}}{\binom{15 + 6 - 1}{15}} = 0.0511
$$

This makes sense in the denominator for the same reason as above. The numerator becomes $\binom{7 + 6 - 1}{7}$ because there are only 7 bagels of freedom (1 sesame bagel is already accounted for in the 1 bagel of each kind)

\item Q64

$$
\frac{\binom{n}{1} \cdot \binom{n - 1}{n - 3} \cdot \frac{n!}{3!} + \binom{n}{2} \cdot \binom{n - 2}{n - 4} \cdot \frac{n!}{2! \cdot 2!}}{n^n}
$$

There are two cases: 

\begin{enumerate}
    \item There is 1 number repeated 3 times. In this case there are $\binom{n}{1}$ ways to pick the number that is repeated. There are $\binom{n - 1}{n - 3}$ ways to pick the rest of the numbers. Lastly, there are $\frac{n!}{3!}$ ways to arrange this sequence. 
    \item There are two numbers repeated 2 times. In this case there are $\binom{n}{2}$ ways to choose the numbers that are repeated twice. There are $\binom{n - 2}{n - 4}$ ways to pick the remaining numbers. Lastly, there are $\frac{n!}{2! \cdot 2!}$ ways to arrange this sequence
\end{enumerate}

So the probability is the total number of possibilities between the two cases over the total number of ways to create a sequence this way, which is $n^n$.

\item 

$$\binom{n + q - 1}{n}$$

This is simply a a question of how you can partition $n$ people among $q$ clubs. 

\end{enumerate}



\end{document}

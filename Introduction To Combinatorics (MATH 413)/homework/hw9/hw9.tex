\documentclass[12pt]{exam}
\usepackage{amsthm}
\usepackage{libertine}
\usepackage[utf8]{inputenc}
\usepackage[margin=1in]{geometry}
\usepackage{amsmath,amssymb}
\usepackage{multicol}
\usepackage[shortlabels]{enumitem}
\usepackage{siunitx}
\usepackage{cancel}
\usepackage{graphicx}
\usepackage{pgfplots}
\usepackage{listings}
\usepackage{tikz}

\usepackage{youngtab}


\pgfplotsset{width=10cm,compat=1.9}
\usepgfplotslibrary{external}
\tikzexternalize

\newcommand{\class}{MATH 413} % This is the name of the course 
\newcommand{\examnum}{Homework 9} % This is the name of the assignment
\newcommand{\examdate}{11/14/2022} % This is the due date
\newcommand{\timelimit}{}


\begin{document}
\pagestyle{plain}
\thispagestyle{empty}

\noindent
\begin{tabular*}{\textwidth}{l @{\extracolsep{\fill}} r @{\extracolsep{6pt}} l}
\textbf{\class} & \textbf{Name:} & \textit{Amit Sawhney}\\ %Your name here instead, obviously 
\textbf{\examnum} &&\\
\textbf{\examdate} &&\\
\end{tabular*}\\
\rule[2ex]{\textwidth}{2pt}
% ---


\begin{enumerate} 

\item Question 1

\begin{proof}
Consider every partition for $n$. Each partition can be extended to be a partition $n + 1$ by adding $1$ to the set of elements in a particular partition. Clearly, this shows that there are at least $p(n)$ partitions for $p(n + 1)$. Additionally, the value $n + 1$ by itself is a valid partition of $n + 1$. Trivially, this could not be a partition of anything less than or equal to $n$, so there must be more partitions of $n + 1$ than $n$. So, $p(n + 1)  > p(n)$.
\end{proof}

\item Question 2

Given an integer $n$ construct the following algorithm to build a self-conjugate with at least 2 parts. There are two cases to construct this algorithm: 

\begin{enumerate}
    \item Case 1: $n$ is odd. Create a partition for $n$ with the value $\frac{n - 1}{2} + 1$ and $\frac{n - 1}{2}$ $1$'s. 
    
    $$\frac{n - 1}{2} + 1 + \frac{n - 1}{2} = n - 1 + 1 = n$$
    
    Additionally this is clearly self conjugate as the Ferrer diagram will be of the form: 
    
    \[\young(~~~,~,~)\]
    
    The first row will be $\frac{n - 1}{2}$ blocks wide and the first column will be $\frac{n - 1}{2}$ blocks tall. This is clearly symmetric under the transpose operation so this is a self-conjugate partition. 
    
    Lastly, because $n > 2$, $$\frac{n - 1}{2} \ge 1$$ So there Ferrer Diagram will always be at least 2 blocks vertically, or in other words it will always have at least 2 parts. 
    
    \item Case 2: $n$ is even. Create a self conjugate partition for $n - 1$ using the algorithm created in Case $1$. Then, replace a single "$1$" with a "$2$". Similarly, 
    
    $$\Bigg ( \frac{ (n - 1) - 1}{2} + 1\Bigg ) + \Bigg ( \frac{(n - 1) - 1}{2} \Bigg ) + 2 = n - 2 +  1 - 1 + 2 = n $$
    
    This creates Ferrer Diagrams of the form:
    
    \[\young(~~~~,~~,~,~)\]
    
    Clearly, this creates the Ferrer Diagram for the odd number below the even number and then adds a box to represent the original even number on a position to maintain symmetry. 
    
\end{enumerate}

\item Question 3

First, we need to construct a generating series for the number of partitions of $n$ in which no part appears exactly once. Clearly, this is represented by:

$$\Big ( (x)^0 + (x)^2 + (x)^3 + (x)^4 + \cdots \Big ) \Big ( (x^2)^0 + (x^2)^2 + (x^2)^3 + \cdots \Big ) \Big ( (x^3)^0 + (x^3)^2 + (x^3)^3 + \cdots \Big )$$
$$= (1 + x^2 + x^3 + \cdots)(1 + (x^2)^2 + (x^2)^3 + \cdots)(1 + (x^3)^2 + (x^3)^3 + \cdots)$$
$$
= \prod_{i = 1}^{\infty} \Big ( 1 + x^{2i} + x^{3i} + x^{4i} + \cdots \Big ) = \prod_{i = 1}^{\infty} \Big (1 + x^{2i}(1 + x^{i} + x^{2i} + x^{3i} \cdots) \Big )
$$
$$
= \prod_{i = 1}^{\infty} \Big (1 + \frac{x^{2i}}{1 - x^{i}} \Big ) = \prod_{i = 1}^{\infty} \Big (\frac{1 - x^{i} + x^{2i}}{1 - x^{i}} \Big ) = \prod_{i = 1}^{\infty} \Big ( \frac{(1 - x^{i} + x^{2i})(1 + x^{i})}{(1 - x^{i})(1 + x^{i})}  \Big )
$$
$$
= \prod_{i = 1}^{\infty} \Big ( \frac{1 + x^{3i}}{1 - x^{2i}} \Big )
$$

Second, we need to construct a generating series for the number of partitions of $n$ with no parts congruent to $1$ or $5 \pmod{6}$. 

This will be based on the normal generating series, except only including terms that are not of this congruence class. 

So the following generating series takes all of the partitions and extends to only be of congruence to $0, 2, 3, 4 \pmod 6$. However, we can rewrite this by taking all of the partitions composed of numbers congruent to $1$ or $5 \pmod{6}$ and divide out all of the partitions, leaving just the partitions with numbers congruent to $0, 2, 3, $ or $4 \pmod{6}$

$$
\prod_{i = 1}^{\infty}
\frac{1}{(1 - x^{6i + 0})(1 - x^{6i + 2}) (1 - x^{6i + 3}) (1 - x^{6i + 4})}
= \prod_{i = 0}^{\infty} \frac{(1 - x^{6i + 1})(1 - x^{6i + 5})}{1 - x^{i}}
$$

Algebraically, we can convert the first expression into the second expression. 

$$
= \prod_{i = 1}^{\infty} \Big ( \frac{1 + x^{3i}}{1 - x^{2i}} \Big )
= \prod_{i = 1}^{\infty} \Bigg ( \frac{(1 + x^{3i})(1 - x^{3i})}{(1 - x^{2i})(1 - x^{3i})} \Bigg )
= \prod_{i = 1}^{\infty} \Bigg ( \frac{(1 - x^{6i})}{(1 - x^{2i})(1 - x^{3i})} \Bigg )
$$
$$
= \prod_{i = 1}^{\infty} \Bigg ( -x^{k} + \frac{1}{1 - x^{k}} \Bigg )
= \prod_{i = 1}^{\infty} \Bigg ( \frac{x^{2i} - x^{i} + 1}{1 - x^{i}} \Bigg ) 
$$

Since, the denominators are equal, we are really just solving: 

$$
\prod_{i = 1}^{\infty} \Big (1 - x^i + x^{2i} \Big ) = \prod_{i = 0}^{\infty} \Big ( (1 - x^{6i + 1})(1 - x^{6i + 5}) \Big )
$$

As per the source, we can realize that he number of partitions of $n$ into parts that are all congruent to $1 \pmod 6$ is equal to the number to partitions $n$ into distinct parts that are all congruent to $1 \pmod 3$ through difference of squares. See as follows: 

$$
\prod_{i = 0}^{\infty} (1 + x^{3i + 1})(1 + x^{3i + 2}) = \frac{\prod_{i = 0}^{\infty} ( 1- x^{6i + 2})(1 - x^{6i + 4})}{\prod_{i = 0}^{\infty}(1 - x^{3i  + 1})( 1- x^{3i + 2})}
$$
$$
\frac{\prod_{i = 0}^{\infty} ( 1- x^{6i + 2})(1 - x^{6i + 4})}{\prod_{i = 0}^{\infty}(1 - x^{6i + 1})(1 - x^{6i + 4})(1 - x^{6i + 2})(1 - x^{6i + 5})}
$$
$$
= \prod_{i = 0}^{\infty} \Bigg ( \frac{1}{(1 - x^{6i + 1})(1 - x^{6i + 5})} \Bigg )
$$

Thus, 

$$
\prod_{i = 0}^{\infty}(1 - x^{6i + 1})(1 - x^{6i + 5}) = \frac{1}{\prod_{i = 0}^{\infty}(1 + x^{3i + 1})(1 + x^{3i + 2})} = \frac{\prod_{i = 1} ( 1 + x^{3i})}{\prod_{i = 1}^{\infty}( 1 + x^{i})}
$$

Lastly, we need to show

$$
\prod_{i = 1}^{\infty} (1 - x^{i} + x^{2i}) = \frac{\prod_{i = 1}^{\infty} ( 1 + x^{3i})}{\prod_{i = 1}^{\infty}( 1 + x^{i})}
$$

So, 

$$
\prod_{i = 1}^{\infty} ( 1 + x^{i})(1 - x^{i} + x^{2i}) = \prod_{i = 1}^{\infty} (1 + x^{3i})
$$
$$
\iff ( 1 + x^{i})(1 - x^{i} + x^{2i}) = (1 + x^{3i})
$$

Which is the sum of cubes identity. So, the generating series are equivalent. 

Citation: https://math.stackexchange.com/questions/3733360/an-application-of-integer-partitions

\item Question 4

\begin{proof}
Consider an integer $n$. Let the partition into distinct parts be written as $$n = d_1 + d_2 + \cdots + d_k$$

Each $d_i$ can be written as $2^j\cdot \ell$ where $j \in \mathbb{Z}$ and $\ell$ is an odd integer. So, $$n = 2^{j_1}\cdot \ell_1 + 2^{j_2} \cdot \ell_2 + \cdots 2^{j_k}\cdot \ell_k$$

We can group the powers of $2$ by their odd counterparts. Such follows as: $$n = (2^{\alpha_1} + 2^{\alpha_2} + \cdots + 2^{\alpha_x}) \cdot \ell_1 + (2^{\beta_1} + 2^{\beta_2} + \cdots + 2^{\beta_y}) \cdot \ell_2 + \cdots $$ where $l_i \in \{1, 3, 5, 7, 9, \dots\}$

Clearly, each $\alpha_i, \beta_j, $ and subsequent grouped powers of $2$ must be unique, otherwise there would a contradiction with the original partition being distinct integers $d_1, d_2, \dots d_k$. 

Let $s$ be the sum of each group of powers of $2$. So, $s_1 = 2^{\alpha_1} + 2^{\alpha_2} + \cdots + 2^{\alpha_x}$, $s_2 = 2^{\beta_1} + 2^{\beta_2} + \cdots + 2^{\beta_y}$, etc.

We can create a partition of $s_1$ $\ell_1$'s, $s_2$ $\ell_2$'s, etc. Thus, we can define a bijective function $f$ that maps partitions with distinct parts to partitions with odd parts using this process. 
\end{proof}

Each distinct partition can clearly only map to one valid odd partition due to the uniqueness and this process is clearly reversible to obtain a distinct partition from an odd partition. Thus, it is bijective.  

\end{enumerate}


\end{document}
